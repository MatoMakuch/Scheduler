\documentclass{article}
\usepackage[utf8]{inputenc}
\usepackage[slovak]{babel}
\usepackage{enumitem}

\begin{document}
	
	\title{Scheduler}
	\author{Martin Makuch}
	\date{\today}
	
	\begin{document}
		
	\maketitle
	
	\section*{Úvod}
	V tejto práci predstavujem výsledky svojho ročníkového projektu s názvom "Scheduler," ktorý bol vyvíjaný pre firmu MicroStep spol. s r. o. Cieľom projektu je navrhnúť a implementovať algoritmus na efektívne plánovanie rozvrhov práce v prostredí typu 'Job Shop'. Tento typ problému je charakterizovaný množinou prác a strojov, kde každá práca pozostáva z postupnosti operácií, ktoré musia byť vykonané na konkrétnych strojoch v stanovenom poradí. Projekt sa zameriava na optimalizáciu časového rozvrhu tak, aby boli všetky práce vykonané v čo najkratšom čase. 
	
	Cieľom projektu je nasadenie v reálnom prostredí výroby v Hríňovej a Partizánskom, kde sa projekt bude testovať. To umožní overiť efektívnosť navrhnutého riešenia v praxi a identifikovať prípadné nedostatky či oblasti na zlepšenie.
	
	\section*{Klasifikácia problému}
	Nech \( m, n \) sú ľubovoľné nezáporné premenné. Potom množina \( M = \{ M_1, M_2, \ldots, M_m \} \) označuje stroje. Množina \( J = \{ J_1, J_2, \ldots, J_n \} \) označuje práce. Pre každú prácu \( J_j \) z množiny \( J \) definujeme čísla: \( m_j \) označujúce počet operácii; \( p_{j1}, p_{j2}, \ldots, p_{jm_j} \) označujúce doby spracovania; \( d_j \) označujúce čas ukončenia práce; \( w_j \) označujúce relatívnu dôležitosť.
	
	\section*{Nastavenie strojov}
	Problém, ktorý riešime je v literatúre označovaný ako 'Job Shop'. Každá práca \( J_j \) z množiny \( J \) pozostáva z usporiadanej \( m_j \)-tice operácii \( ( O_{j1}, O_{j2}, \ldots, O_{jm_j} ) \), kde číslo \( m_j \) reprezentuje počet operácii, z ktorých práca \( J_j \) pozostáva, a kde sa operácia \( O_{ji} \) musí v poradí vykonať na stroji \( \mu_{ji} \) za čas \( p_{ji} \). Pritom pre \( i = 1, 2, \ldots, m_j \) platí, že \( \mu_{i-1} \neq \mu_i \).
	
	\section*{Charakteristika prác}
	V našom prostredí zakazujeme prerušovanie vykonávania jednotlivých operácii. Žiadne ďalšie obmedzenia nateraz nekladieme.
	
	\section*{Cieľ}
	Cieľom je naplánovať rozvrh pre optimálne vykonanie prác (za optimálne riešenie budeme považovať také, ktoré v najkratšom čase vykoná všetky práce).
	
	\section*{Brute-force algoritmus}
	Keďže každá práce pozostáva z nejakej postupnosti operácii, kde každá z nich musí byť vykonaná na práve jednom stroji - môžeme usporiadať práce podľa toho na akom stroji sa majú vykonať.
	
	Zjavne v každom rozvrhu platí, že ak sa ľubovoľné dve operácie musia vykonať na jednom stroji, tak niektorá z nich sa musí vykonať skôr. Výmenou poradia ľubovoľných dvoch operácii teda dostaneme nový rozvrh. (Výmeny poradia jednotlivých strojov uvažovať nebudeme vzhľadom k tomu, že stroje pracujú nezávisle.)
	
	Vo všeobecnosti by sme počet všetkých rôznych rozvrhov (teda takých, ktoré sa líšia poradím nejakých dvoch operácii prislúchajúcich jednému stroju) mohli zapísať následovne: nech \( [M_i] \) označuje triedu ekvivalencie na množine všetkých operácii \( O = O_1 \cup \ldots \cup O_n \); potom \( \prod_{i=1}^{m} |[M_i]|! \) vyjadruje hľadaný počet.
	
	Výsledný rozvrh aj s časmi vykonania jednotlivých operácii zostrojíme následovne:
	\begin{enumerate}[label=\arabic*.]
		\item rozdelíme operácie do radov podľa strojov, na ktorých sa majú vykonať, v poradí, v ktorých sa majú vykonať;
		\item opakujeme v cykle:
		\begin{enumerate}[label*=\arabic*.]
			\item počínajúc ľubovoľným strojom iterujeme všetky stroje; môžu nastať dva prípady:
			\begin{enumerate}[label*=\arabic*.]
				\item operácia na začiatku rady sa môže vykonať t.j. všetky predchádzajúce operácie sa už vykonali:
				\begin{enumerate}[label*=\arabic*]
					\item vyberieme opráciu z rady a naplánujeme ju,
					\item pokračujeme na ďalšou opráciou v rade;
				\end{enumerate}
				\item operácia na začaitku rady sa ešte vykonať nemôže t.j. niektorá z predchádzajúcich operácii sa ešte nevykonala,
				\begin{enumerate}[label*=\arabic*]
					\item prejdeme na ďalší stroj;
				\end{enumerate}
			\end{enumerate}
			\item ak sa nenaplánovala žiadna operácia, ukončíme cyklus;
		\end{enumerate}
		\item po skončení cyklu môžu nastať dva prípady:
		\begin{enumerate}[label*=\arabic*]
			\item všetky operácie sú naplánované, vrátime maximum z časov ukončenia všetkých operácii;
			\item nejaká operácia nie je naplánovaná; vrátime nekonečno;
		\end{enumerate}
	\end{enumerate}

	Algoritmu postupne prechádza cez rady operácii prislúchajúce jednotlivým strojom a operáciu vyberie (naplánujeme) ak operácia, ktorá jej má predchádzať je už naplánovaná. 
	
	(Stačí si pri tom pamätať iba poslednú naplánovanú operáciu pre každú prácu.) Počiatočný čas vykonania operácie sa vypočíta ako maximum z počiatočného času vykonania poslednej operácie zväčšeného o dĺžku vykonania tej operácie a počiatočného času poslednej operácie vykonanej na danom stroji zväčšeného o dĺžku vykonania tej operácie. (Pre druhé uvedené nám stačí udržiavať pre každý stroj čas ukončenia poslednej operácie na ňom.)
	
	\section*{Nedostatky}
	S uvažovaným modelom sa síce pracuje príjmne, pre jeho jednoduchosť, je však v praxi len sotva využiteľný. 
	
	V ďaľších sekciách si predstavíme iné, všeobecnejšie, spôsoby pohľadu na operácie. Potešujúce je, že mnohé myšlienky, použité pre tento model, budeme vedieť aplikovať aj na všeobecenjšie verzie problému.
	
	\section*{Zovšeobecnenie}
	Máme výrobnú linku obsahujúcu štyri pracoviská: rezanie, vŕtanie, ohýbanie a skladanie. Proces výroby prebieha nasledovne:
	
	\begin{enumerate}
		\item Na \textbf{prvom pracovisku} (rezanie) vyrežeme z kusu typu \textbf{A} dva kusy typu \textbf{B}.
		\item \textbf{Druhé pracovisko} (vŕtanie) spracuje jeden z kusov typu \textbf{B}, čím vznikne kus typu \textbf{B'}.
		\item \textbf{Tretie pracovisko} (ohýbanie) spracuje druhý kus typu \textbf{B}, čím vznikne kus typu \textbf{B''}.
		\item Na \textbf{štvrtom pracovisku} (skladanie) sa z kusov \textbf{B'} a \textbf{B''} zostaví jeden kus typu \textbf{C}.
	\end{enumerate}
	
	Malo by byť vidieť, že takýto problém nevieme v základnom modeli vyjadriť. konkrétne nám robí problém vyjdariť fakt, že niektoré joby, môžu medzi sebou zdieľať operácie.
	
	\section*{Úpravený brute-force algoritmus}
	Pri priamočiarej úprave reprezentácie následností operácii ako orientovaného acyklického grafu sa pôvodný algoritmus zmení iba v bode, kedy zisťuje, či je možné naplánovať ďalšiu opráciu. O upravenom algoritmne potom vieme argumentovať o jeho správnosti analogicky.
	
	\subsection*{Pozorovanie}
	Upravený algoritmus, ako sme aj ukázali, funguje naďalej. Otázka je ako by sme vedeli znížiť počet všetkých vyskúšaných rozvrhov?
	
	(...)
	
	\section*{Operácie ako vstupno-výstupné relácie}
	Ďalšie zovšeobecnenie, ktoré vieme spraviť je zmena pohľadu na operácie ako na čisto matematický objekt. Toto zovšeobecenenie sa môže ukázať ako kľúčové pri nasadzovaní techník lineárneho programovania.
	
	(...)
	
\end{document}
